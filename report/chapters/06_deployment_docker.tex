% ────────────────────────────────────────────────────────────────────────────
\chapter{Deployment e Containerizzazione Docker}
% ────────────────────────────────────────────────────────────────────────────

L'intero sistema è definito in un singolo file \texttt{docker-compose.yml} che descrive tutti i microservizi e le relative dipendenze. La scelta di un unico Compose file garantisce la riproducibilità dell'ambiente con un singolo comando \texttt{docker compose up}, eliminando qualsiasi configurazione manuale.

\section{Servizi Definiti nel Docker Compose}

\begin{table}[H]
\centering
\footnotesize
\begin{tabularx}{\textwidth}{L{2.2cm} L{4cm} C{1.5cm} L{2.5cm} L{3cm}}
\toprule
\textbf{Container} & \textbf{Immagine} & \textbf{Porte} & \textbf{Volumi} & \textbf{Depends On} \\
\midrule
loki & grafana/loki:3.5.0 & 3100 & \texttt{./loki/} & --- \\
tempo & grafana/tempo:2.9.0 & 3200 & \texttt{./tempo/} & --- \\
otel-collector & otel/opentelemetry-\newline collector-contrib:0.133.0 & 4317, 4318 & \texttt{./otel-collector/} & loki, tempo \\
prometheus & prom/prometheus:v2.53.0 & 9090 & \texttt{./prometheus/} & --- \\
grafana & grafana/grafana:12.2 & 8080$\to$3000 & \texttt{./grafana/} & loki, tempo, prometheus \\
n8n & docker.n8n.io/\newline n8nio/n8n & 5678 & \texttt{./Data\_N8N} & redis, ollama, otel-collector \\
redis & redis:alpine & 6379 & \texttt{./Data\_Redis} & --- \\
ollama & ollama/ollama & 11434 & \texttt{./Data\_Ollama} & --- \\
\bottomrule
\end{tabularx}
\caption{Dettaglio completo dei servizi nel Docker Compose}
\end{table}

\section{Rete Docker, Comunicazione e Dipendenze}

Tutti i container sono connessi a un'unica rete Docker bridge denominata \texttt{rete\_unica}.

\begin{technicalbox}{Dettaglio Tecnico: Topologia di Rete}
All'interno della rete \texttt{rete\_unica}, ogni container è raggiungibile dagli altri tramite il \textbf{nome del servizio} definito nel Docker Compose. Ad esempio:
\begin{itemize}
  \item n8n raggiunge Redis come \texttt{redis:6379} (senza passare dall'host).
  \item n8n raggiunge Ollama come \texttt{ollama:11434}.
  \item n8n invia telemetria OTLP a \texttt{otel-collector:4318}.
  \item L'OTel Collector esporta i log a \texttt{loki:3100/otlp}.
  \item L'OTel Collector esporta le tracce a \texttt{tempo:4317}.
  \item Grafana interroga \texttt{loki:3100}, \texttt{tempo:3200} e \texttt{prometheus:9090} tramite provisioning automatico.
\end{itemize}

Tutte le porte dei servizi sono mappate sull'host per consentire l'accesso diretto durante lo sviluppo e il debugging. In un ambiente di produzione, solo le porte strettamente necessarie all'accesso esterno (5678 per il webhook n8n, 8080 per la UI Grafana) andrebbero esposte.
\end{technicalbox}

\subsection{Catena di Dipendenze e Ordine di Avvio}

Il Docker Compose definisce dipendenze esplicite tramite \texttt{depends\_on}:

\begin{itemize}
  \item \textbf{n8n} $\to$ \texttt{redis}, \texttt{ollama}, \texttt{otel-collector}: n8n non può operare senza lo stato conversazionale (Redis), il motore di inferenza LLM (Ollama) e il collettore di telemetria.
  \item \textbf{otel-collector} $\to$ \texttt{loki}, \texttt{tempo}: Il collettore necessita dei backend di log e tracce attivi per esportare i segnali.
  \item \textbf{grafana} $\to$ \texttt{loki}, \texttt{tempo}, \texttt{prometheus}: I datasource devono essere raggiungibili per il provisioning automatico.
\end{itemize}

\begin{warningbox}{Nota Importante: Ordine di Avvio}
\textit{Le dipendenze \texttt{depends\_on} di Docker Compose garantiscono l'ordine di avvio ma non la readiness dei servizi. In un ambiente di produzione, sarebbe necessario aggiungere \texttt{healthcheck} e \texttt{depends\_on: condition: service\_healthy} per garantire che i servizi upstream siano effettivamente pronti prima di avviare quelli downstream.}
\end{warningbox}

\section{Volumi e Persistenza dei Dati}

Il sistema utilizza \textbf{bind mounts} per la persistenza dei dati e la configurazione:

\begin{table}[H]
\centering
\footnotesize
\begin{tabularx}{\textwidth}{L{2.5cm} L{4.5cm} L{6cm}}
\toprule
\textbf{Servizio} & \textbf{Volume Host} & \textbf{Scopo} \\
\midrule
n8n & \texttt{./Data\_N8N} & Persistenza workflow, credenziali, configurazione n8n \\
redis & \texttt{./Data\_Redis} & Dump RDB per persistenza dello stato \\
ollama & \texttt{./Data\_Ollama} & Cache dei modelli LLM scaricati \\
loki & \texttt{./loki/loki-config.yml} (ro) & Configurazione di Loki (read-only) \\
loki & \texttt{./.runtime/data/loki} & Storage dei chunk di log \\
tempo & \texttt{./tempo/tempo-config.yml} (ro) & Configurazione di Tempo (read-only) \\
tempo & \texttt{./.runtime/data/tempo} & Storage dei blocchi di tracce \\
otel-collector & \texttt{./otel-collector/collector.yaml} (ro) & Configurazione del collettore \\
prometheus & \texttt{./prometheus/prometheus.yml} (ro) & Configurazione di scraping \\
grafana & \texttt{./grafana/provisioning/} (ro) & Datasource e dashboard auto-configurati \\
grafana & \texttt{./.runtime/data/grafana} & Persistenza dashboard e preferenze \\
\bottomrule
\end{tabularx}
\caption{Mappatura completa dei volumi Docker}
\end{table}

I file di configurazione sono montati come \textbf{read-only} (\texttt{:ro}) per prevenire modifiche accidentali da parte dei container.

\section{Variabili d'Ambiente Chiave}

\subsection{n8n}

\begin{lstlisting}[language=bash]
# Timezone
GENERIC_TIMEZONE=Europe/Rome
TZ=Europe/Rome

# Integrazione Osservabilita
OTEL_COLLECTOR_URL=http://otel-collector:4318

# Timeout esecuzione (-1 = nessun timeout)
N8N_EXECUTION_TIMEOUT=-1
\end{lstlisting}

La variabile \texttt{N8N\_EXECUTION\_TIMEOUT=-1} disabilita il timeout di esecuzione dei workflow, necessario perché la generazione del modello architetturale (catena DDD Analyst $\to$ JSON Coder) può richiedere diversi minuti con inferenza locale su modelli da 14 miliardi di parametri.

\subsection{Grafana}

\begin{lstlisting}[language=bash]
GF_SECURITY_ADMIN_USER=admin
GF_SECURITY_ADMIN_PASSWORD=admin
GF_SERVER_ROOT_URL=http://localhost:8080
GF_EXPLORE_ENABLED=true
GF_AUTH_ANONYMOUS_ENABLED=true
GF_AUTH_ANONYMOUS_ORG_ROLE=Admin
\end{lstlisting}

L'accesso anonimo con ruolo Admin è abilitato per ambiente di sviluppo locale, eliminando la necessità di autenticazione. In un ambiente di produzione, questa configurazione andrebbe rimossa.

\section{Politiche di Restart}

Tutti i servizi dello stack di osservabilità sono configurati con \texttt{restart: unless-stopped}, garantendo il riavvio automatico dopo un crash o un riavvio del sistema host. Il servizio n8n non possiede questa politica nel Compose attuale, ma dipende dai servizi upstream che la implementano.
