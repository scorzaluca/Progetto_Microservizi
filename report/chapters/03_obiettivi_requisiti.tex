% ────────────────────────────────────────────────────────────────────────────
\chapter{Obiettivi e Requisiti del Sistema}
% ────────────────────────────────────────────────────────────────────────────

\section{Obiettivi Strategici}

Il DIM si propone di automatizzare e standardizzare la fase di Domain Discovery, operando come primo agente nella pipeline di progettazione architetturale. Gli obiettivi si articolano in quattro direttrici:

\begin{enumerate}
  \item \textbf{Automatizzare la raccolta dei requisiti di dominio} attraverso un'intervista strutturata in fasi progressive, capace di guidare uno stakeholder di business nella definizione completa del dominio applicativo senza richiedere competenze DDD pregresse.
  
  \item \textbf{Generare un modello di dominio formale e machine-readable} a partire dalla conversazione con lo stakeholder, producendo un artefatto strutturato direttamente consumabile dagli agenti downstream della pipeline.
  
  \item \textbf{Supportare il raffinamento iterativo del modello} consentendo allo stakeholder di rivedere, chiarire e modificare l'architettura prodotta attraverso un ciclo di revisione conversazionale con garanzie contro modifiche involontarie.
  
  \item \textbf{Garantire osservabilità end-to-end} dell'intero sistema, fornendo visibilità su log e metriche di tutti i componenti per facilitare il debugging e il monitoraggio operativo.
\end{enumerate}

\section{Requisiti Funzionali}

\subsection{RF-01: Pipeline di Intervista a 5 Fasi}

Il sistema deve implementare cinque agenti LLM specializzati, ciascuno focalizzato su un pillar architetturale del DDD, che conducono sequenzialmente l'intervista. Ogni agente:
\begin{itemize}
  \item Possiede un system prompt rigoroso che ne delimita il focus tematico.
  \item Interagisce esclusivamente in italiano, traducendo il gergo DDD in scenari di business.
  \item Pone una sola domanda alla volta, con drill-down obbligatorio su risposte vaghe.
  \item Non decide autonomamente la fine della propria fase: la transizione è governata esclusivamente dall'Orchestrator (RF-02)
\end{itemize}

\subsection{RF-02: Validazione Automatica di Completezza}

Un agente Orchestrator analizza lo storico conversazionale al termine di ogni fase e determina se le informazioni raccolte sono sufficienti per procedere alla fase successiva. La validazione si basa su una checklist strutturata per fase (es. per la Fase~1: Core Domain identificato, Generic Subdomains listate, Supporting Subdomains specificate).

\subsection{RF-03: Routing Intelligente}

Il sistema deve instradare automaticamente ogni messaggio dell'utente al flusso corretto:
\begin{itemize}
  \item Se il modello architetturale non è ancora stato generato, il messaggio viene inoltrato al flusso di Discovery.
  \item Se il modello è già disponibile, il messaggio viene inoltrato al flusso di Refinement.
\end{itemize}

\subsection{RF-04: Generazione Architetturale}

Al completamento dell'intervista, il sistema deve generare automaticamente un documento architetturale strutturato in cinque pilastri (Strategic Analysis, Boundary Definition, EDA Integration, Tactical Patterns, Technical Excellence) e la sua rappresentazione JSON formale con schema predefinito. La generazione deve operare con politica Zero-Loss, preservando ogni dettaglio tecnico emerso durante l'intervista senza riassumere né omettere informazioni.

\subsection{RF-05: Raffinamento del Modello}

Il sistema deve supportare due modalità operative post-generazione, selezionabili dall'utente:
\begin{itemize}
  \item \textbf{Modalità Spiegazione:} L'utente può porre domande sull'architettura corrente e ricevere risposte contestuali basate sul modello generato, senza che vengano proposte modifiche.
  \item \textbf{Modalità Modifica:} L'utente può richiedere modifiche strutturali al modello. Il sistema produce una bozza aggiornata che l'utente può confermare o scartare.
\end{itemize}

\subsection{RF-06: Gestione del Ciclo di Vita delle Bozze}

Il sistema deve gestire il ciclo di vita delle bozze di modifica con un meccanismo a doppia conferma:
\begin{itemize}
  \item \textbf{Conferma:} La bozza viene promossa a modello corrente.
  \item \textbf{Rifiuto:} La bozza viene eliminata e il modello corrente rimane invariato.
  \item Entrambe le azioni richiedono una seconda conferma esplicita per prevenire operazioni accidentali.
\end{itemize}

\subsection{RF-07: Frontend Chat}

Interfaccia web single-page con:
\begin{itemize}
  \item Chat full-screen con layout responsive.
  \item Sessioni persistenti che consentano la ripresa di conversazioni interrotte.
  \item Rendering differenziato per messaggi testuali, blocchi JSON con sintassi evidenziata, e contenuti Markdown.
  \item Feedback visuale contestuale che distingua le diverse fasi operative (discovery, spiegazione, modifica).
\end{itemize}

\subsection{RF-08: Stato Conversazionale Persistente}

Il sistema deve mantenere uno stato conversazionale persistente per ogni sessione, includendo: la fase corrente dell'intervista, lo storico completo dei messaggi, il modello architetturale generato e l'eventuale bozza di modifica in attesa di conferma. Lo stato deve sopravvivere al riavvio del browser e consentire la ripresa della sessione entro un tempo ragionevole.

\section{Requisiti Non Funzionali}

\begin{itemize}
  \item \textbf{RNF-01 -- Osservabilità:} Raccolta centralizzata di log (Loki) e metriche (Prometheus) tramite OpenTelemetry Collector, con datasource Grafana preconfigurati.
  \item \textbf{RNF-02 -- Privacy e Indipendenza:} Inferenza LLM completamente locale tramite Ollama. Nessun dato inviato a servizi cloud esterni.
  \item \textbf{RNF-03 -- Portabilità:} Containerizzazione completa tramite Docker Compose per ambiente self-contained e riproducibile con un singolo \texttt{docker compose up}.
  \item \textbf{RNF-04 -- Modularità:} Ogni workflow n8n è un componente autonomo richiamabile indipendentemente, con interfacce well-defined (input/output via webhook e Redis).
  \item \textbf{RNF-05 -- Resilienza Sessionale:} Lo stato conversazionale in Redis con TTL di 24 ore permette la ripresa di sessioni interrotte senza perdita di dati.
  \item \textbf{RNF-06 -- Estensibilità:} L'architettura a workflow consente l'aggiunta di nuove fasi, nuovi agenti o nuovi modelli LLM senza modifiche al codice applicativo.
\end{itemize}
