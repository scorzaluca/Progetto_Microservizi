% ────────────────────────────────────────────────────────────────────────────
\chapter{Analisi dello Scenario e Business Case}
% ────────────────────────────────────────────────────────────────────────────

\section{Razionale del Progetto}

In un contesto aziendale reale, la definizione del dominio applicativo e la sua suddivisione in sottodomini funzionali rappresenta il primo passo critico nella progettazione di un sistema basato su microservizi. Questo processo, tipicamente condotto manualmente da architetti esperti attraverso sessioni di Event Storming e interviste con gli stakeholder, è lungo, soggetto a bias individuali e difficilmente ripetibile in modo sistematico.

Il Domain-Driven Design (DDD) fornisce un framework concettuale potente per affrontare questa sfida, ma la sua applicazione richiede un livello di competenza specialistica che non è sempre disponibile nei team di sviluppo. Il DIM nasce per colmare questo gap: automatizzare e strutturare la fase fondamentale di Domain Discovery, rendendo accessibile a qualunque stakeholder di business la produzione di un modello architetturale formale conforme ai principi DDD.

\section{Analisi dei Pain Points}

Lo scenario tipico di una fase di discovery manuale presenta criticità strutturali che il DIM intende risolvere:

\begin{enumerate}
  \item \textbf{Raccolta non strutturata:} Le interviste con gli stakeholder producono note informali, appunti sparsi e conoscenza implicita che non viene formalizzata in un modello coerente. La trascrizione manuale introduce ulteriori perdite informative.
  
  \item \textbf{Assenza di copertura sistematica:} Senza una guida strutturata per fasi, l'architetto rischia di trascurare aspetti critici come i pattern di gestione dei fallimenti (Saga), i modelli di consistenza (Strong vs. Eventual), o le esigenze di integrazione con sistemi legacy tramite Anticorruption Layer.
  
  \item \textbf{Soggettività dell'analista:} Il risultato di una sessione DDD tradizionale varia significativamente in funzione di chi la conduce. Un architetto con esperienza prevalente in sistemi monolitici, ad esempio, tenderà inconsciamente ad accorpare responsabilità in pochi bounded context di grandi dimensioni, introducendo bias strutturali nel modello di dominio. Non esistendo un protocollo di indagine standardizzato, la qualità dell'output dipende dall'intuizione e dal background individuale del'analista, rendendo il processo difficilmente ripetibile.
  
  \item \textbf{Non ripetibilità:} Due sessioni di discovery sullo stesso progetto, condotte da architetti diversi o dallo stesso architetto in momenti diversi, possono produrre modelli significativamente diversi. Non esiste un benchmark di completezza.
  
  \item \textbf{Assenza di output machine-readable:} Anche quando la sessione produce risultati di qualità, la documentazione rimane in formato testuale non strutturato, richiedendo un ulteriore passaggio manuale per la formalizzazione in modelli consumabili da strumenti di progettazione.
\end{enumerate}

\section{Rischi e Impatto Operativo}

Un modello di dominio incompleto o mal definito ha ripercussioni a cascata sull'intera architettura a microservizi:

\begin{itemize}
  \item \textbf{Bounded Context errati:} Responsabilità sovrapposte tra microservizi, con conseguente accoppiamento stretto e difficoltà di deploy indipendente.
  \item \textbf{Modello event-driven incompleto:} L'omissione o la ridondanza di eventi durante la fase di discovery produce, in produzione, eventi che nessun servizio è in grado di elaborare, compromettendo la consistenza dei dati e rendendo i guasti difficili da diagnosticare.
  \item \textbf{Ubiquitous Language inconsistente:} Ambiguità terminologiche che si propagano dal modello di dominio al codice sorgente, generando bug semantici difficili da diagnosticare.
  \item \textbf{Pattern tattici inappropriati:} Scelta di Choreography dove sarebbe necessaria un'Orchestration (o viceversa), con impatto sulla tracciabilità dei failure e sulla complessità del debugging.
\end{itemize}

Il DIM mitiga questi rischi attraverso la sistematicità della pipeline a cinque fasi, la persistenza dello stato conversazionale in Redis, la validazione automatica della completezza tramite il Project Controller, e la generazione automatica di un modello JSON strutturato, machine-readable e revisionabile.

\section{Business Case}

Il costo reale di una sessione di Domain Discovery tradizionale è spesso sottostimato. Un workshop di Event Storming richiede mediamente \textbf{uno o due giorni} di lavoro intensivo (circa 10--14 ore effettive), con la partecipazione simultanea di sviluppatori, product manager, domain expert e architetti. A livello di mercato, il costo di una sessione facilitata da un consulente specializzato parte da circa \textbf{500 EUR/persona/giorno}.

Il costo più significativo, tuttavia, non è quello diretto ma quello \textbf{nascosto}: un modello di dominio errato o incompleto si propaga a cascata nell'architettura, generando debito tecnico che emergerà settimane o mesi dopo sotto forma di bounded context da ristrutturare, eventi da ridefinire e servizi da riscrivere. In contesti enterprise, il costo di correzione di errori architetturali in produzione può superare di ordini di grandezza quello della sessione di discovery iniziale.

Il DIM comprime l'intero ciclo di discovery in una \textbf{singola sessione conversazionale asincrona}, eliminando la necessità di coordinare simultaneamente più partecipanti e producendo un output strutturato e machine-readable in tempo reale. Lo stakeholder di business può interagire con il sistema in autonomia, nei propri tempi, senza dipendere dalla disponibilità di un architetto specializzato.

\section{Confronto con le Alternative Esistenti}

Gli strumenti attualmente disponibili per supportare il processo di Domain Discovery si collocano in tre categorie, ciascuna con limitazioni significative che il DIM intende superare:

\begin{enumerate}
  \item \textbf{Whiteboard collaborativi generici} (Miro, Mural, FigJam): Forniscono un canvas digitale con sticky notes e strumenti di collaborazione in tempo reale, ma non offrono alcuna struttura DDD nativa. La responsabilità di guidare la sessione, formulare le domande giuste e garantire la completezza dell'analisi ricade interamente sull'architetto umano. L'output rimane un artefatto visuale non strutturato, non consumabile automaticamente da strumenti downstream.

  \item \textbf{Tool DDD specializzati} (Context Mapper, Qlerify, Domo): Offrono primitive di modellazione specifiche per il DDD --- bounded context, context mapping, aggregate --- ma presuppongono che l'utente conosca già il dominio e sappia come modellarlo. Sono strumenti di \textit{formalizzazione}, non di \textit{scoperta}: supportano la fase di modellazione ma non quella di raccolta e scoperta dei requisiti.

  \item \textbf{LLM general-purpose} (ChatGPT, Claude): Possono rispondere a domande sul DDD e suggerire strutture architetturali, ma operano in modo reattivo e non strutturato. Non garantiscono copertura sistematica delle cinque aree del DDD strategico, non mantengono stato conversazionale tra le sessioni, e soprattutto trasmettono i dati di dominio --- potenzialmente riservati --- a servizi cloud di terze parti.
\end{enumerate}

Il DIM si differenzia combinando le tre capacità che nessuna alternativa offre congiuntamente: \textbf{raccolta guidata dei requisiti} (intervista strutturata a cinque fasi con drill-down), \textbf{validazione automatica della completezza} (Project Controller), e \textbf{generazione di output machine-readable} (JSON architetturale conforme a schema predefinito), il tutto operando \textbf{interamente in locale} senza trasmissione di dati a servizi esterni.

