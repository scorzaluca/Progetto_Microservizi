% ────────────────────────────────────────────────────────────────────────────
\chapter{Frontend Chat}
% ────────────────────────────────────────────────────────────────────────────

Il frontend è un'interfaccia web single-page (\texttt{frontend/index.html}) che consente allo stakeholder di interagire con il DIM attraverso una chat full-screen. L'intera interfaccia è contenuta in un unico file HTML con CSS e JavaScript inline, senza dipendenze esterne eccetto la libreria \texttt{marked.js} per il rendering Markdown.

\begin{figure}[H]
    \centering
    \includegraphics[width=\textwidth]{figures/frontend_chat.png}
    \caption{Interfaccia della chat frontend}
    \label{fig:frontend_chat}
\end{figure}

\section{Layout e Design}

L'interfaccia adotta un layout orizzontale split-view:
\begin{itemize}
  \item \textbf{Lato sinistro (70\%):} Area dei messaggi (\texttt{\#chat-history}) con scrolling verticale, bolle di chat differenziate per colore (blu per l'utente, grigio per l'agente).
  \item \textbf{Lato destro (30\%):} Area di input con textarea multi-linea e pulsante ``Invia''.
  \item \textbf{Header:} Barra superiore blu con titolo ``Intervista DDD'' e pulsante ``Nuova Chat'' per il reset della sessione.
\end{itemize}


\section{Gestione delle Sessioni}

Ogni sessione è identificata da un \texttt{sessionId} univoco generato lato client:

\begin{lstlisting}[language=Java]
let sessionId = localStorage.getItem('chat_session_id');
if (!sessionId) {
    sessionId = 'sess_' + Math.random().toString(36).substr(2, 9);
    localStorage.setItem('chat_session_id', sessionId);
}
\end{lstlisting}

Il \texttt{sessionId} è persistito in \texttt{localStorage}, garantendo la continuità della sessione anche dopo il refresh della pagina. Il pulsante ``Nuova Chat'' rimuove la chiave e ricarica la pagina, generando un nuovo ID.

\section{Rendering Differenziato dei Messaggi}

Il frontend distingue tre tipologie di contenuto per i messaggi dell'agente, ciascuna con un rendering dedicato:

\begin{itemize}
  \item \textbf{Messaggi Testuali:} Bolle di chat standard con label ``Agente DDD''.
  \item \textbf{Blocchi JSON Architetturali:} Visualizzazione stile terminale con formattazione pretty-print e pulsante ``Copia JSON''.
  \item \textbf{Blocchi Markdown:} Documenti architetturali renderizzati in HTML tramite \texttt{marked.js}, con label differenziata (``Documento Architetturale'' per il modello finale, ``Agente Modifier'' per le bozze di modifica).
\end{itemize}

\section{Feedback Visuale e Animazioni}

Il frontend implementa diverse animazioni per fornire feedback all'utente durante le operazioni asincrone:

\begin{table}[H]
\centering
\footnotesize
\begin{tabularx}{\textwidth}{L{3.5cm} L{3cm} L{6.5cm}}
\toprule
\textbf{Contesto} & \textbf{Animazione} & \textbf{Descrizione} \\
\midrule
Discovery / Explain & Typing indicator & Tre puntini blu che rimbalzano (bounce animation) \\
Generazione modello & Slide right-to-left & Testo ``Generazione architettura in corso...'' blu che scorre \\
Modifica & Slide left-to-right & Testo ``Analizzando la richiesta...'' arancione che scorre \\
\bottomrule
\end{tabularx}
\caption{Animazioni di feedback nel frontend}
\end{table}

\section{Pulsanti Inline e Selezione Modalità}

Dopo la generazione del modello architetturale, il frontend presenta due pulsanti inline nell'area chat:

\begin{itemize}
  \item \textbf{``Chiedi Spiegazione'':} Attiva la modalità \texttt{explain}. L'area di input viene sbloccata con placeholder ``Scrivi la tua domanda...''.
  \item \textbf{``Richiedi Modifica'':} Attiva la modalità \texttt{modify}. L'area di input viene sbloccata con placeholder ``Descrivi la modifica strutturale che desideri...''.
\end{itemize}

Durante la visualizzazione dei pulsanti, l'area di input è \textbf{bloccata} con il messaggio ``Seleziona un'azione dalla chat prima di scrivere...'', forzando l'utente a una scelta esplicita.

\section{Meccanismo di Draft con Double-Check}

Quando l'utente riceve una bozza di modifica (modalità Modify), il frontend mostra pulsanti ``Conferma Modifica'' (verde) e ``Rifiuta Modifica'' (rosso). Al click:

\begin{enumerate}
  \item I pulsanti originali vengono nascosti.
  \item Appare un messaggio di conferma contestuale:
  \begin{itemize}
    \item Per la conferma: ``Sei sicuro che vuoi che questa sia la modifica che vuoi?''
    \item Per il rifiuto: Un messaggio più lungo che avverte della perdita della proposta.
  \end{itemize}
  \item Pulsanti ``Sì'' / ``No'' per la conferma finale.
  \item Il ``No'' ripristina i pulsanti originali; il ``Sì'' invia la decisione al Draft Decision Workflow.
\end{enumerate}

Questo meccanismo a doppia conferma previene click accidentali che potrebbero sovrascrivere l'architettura corrente.

\section{Comunicazione con il Backend}

\begin{table}[H]
\centering
\footnotesize
\begin{tabularx}{\textwidth}{L{3cm} C{1.3cm} X L{4cm}}
\toprule
\textbf{Azione} & \textbf{Metodo} & \textbf{Endpoint} & \textbf{Payload} \\
\midrule
Invio messaggio & POST & \texttt{/webhook/chat-locale} & \texttt{message, \allowbreak sessionId, \allowbreak action} \\
Polling modello & GET & \texttt{/webhook/check-status-locale} & \texttt{sessionId} (query param) \\
Draft Decision & POST & \texttt{/webhook/draft-decision-locale} & \texttt{sessionId, \allowbreak action} \\
\bottomrule
\end{tabularx}
\caption{Endpoint di comunicazione Frontend $\to$ n8n}
\end{table}

Il polling viene avviato automaticamente quando il frontend riceve dal server una risposta con \texttt{status: generating}. Questo avviene immediatamente dopo che l'Orchestrator, al termine della conversazione in Fase~5, emette un verdetto di \textbf{PASS}, autorizzando l'avvio della catena di generazione asincrona. L'intervallo di interrogazione è impostato a 3 secondi.

\section{Gestione degli Errori}

Il frontend gestisce gli errori a più livelli:
\begin{itemize}
  \item \textbf{Errore di rete:} Messaggio ``Errore di comunicazione con il server.'' nella chat.
  \item \textbf{Risposta vuota o 500:} Messaggio ``Il server ha risposto con un errore (500) o un contenuto vuoto. Controlla i workflow su n8n.''.
  \item \textbf{Fallimento modifica:} Se il Draft Decision restituisce \texttt{status: failed}, il frontend mostra ``Ops! I controlli di sicurezza interni hanno bloccato questa modifica.''.
  \item In tutti i casi di errore, l'input viene ri-abilitato per permettere all'utente di riprovare.
\end{itemize}
