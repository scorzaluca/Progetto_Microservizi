% ────────────────────────────────────────────────────────────────────────────
\chapter{Conclusioni e Sviluppi Futuri}
% ────────────────────────────────────────────────────────────────────────────

\section{Sintesi dei Risultati}

Il Domain Interviewer \& Modeler dimostra la fattibilità di un approccio completamente automatizzato alla scoperta del dominio e alla progettazione architetturale DDD tramite agenti LLM specializzati. I risultati principali possono essere sintetizzati nei seguenti punti:

\begin{enumerate}
  \item \textbf{Pipeline Multi-Agente Funzionante:} Cinque agenti intervistatori specializzati, un Project Controller di validazione e una catena di generazione (DDD Analyst + JSON Coder) cooperano per trasformare una conversazione in linguaggio naturale in un modello architetturale JSON strutturato e machine-readable.

  \item \textbf{System Prompt Engineering Efficace:} La struttura a quattro sezioni (Role, Output Rules, Anti-Bleeding, Discovery Flow) con vincoli di inibizione espliciti si è dimostrata efficace nel mantenere gli agenti focalizzati sul proprio ambito tematico, riducendo significativamente il fenomeno di bleeding tra fasi.

  \item \textbf{Orchestrazione Senza Codice Custom:} L'intera logica applicativa --- routing, branching condizionale, invocazione LLM, gestione stato, validazione --- è implementata come workflow n8n, senza una singola riga di codice backend tradizionale. Questo approccio ha accelerato lo sviluppo e facilitato il debugging grazie alla visibilità nativa dell'engine di workflow.

  \item \textbf{Privacy-by-Design Praticabile:} L'inferenza locale tramite Ollama, pur con i compromessi di latenza documentati nel Capitolo~11, dimostra che un sistema di discovery architetturale può operare in totale autonomia senza dipendenze cloud, rendendo il DIM adatto a contesti con vincoli di riservatezza.

  \item \textbf{Raffinamento Interattivo Sicuro:} Il meccanismo di Draft/Confirm/Discard con double-check previene efficacemente le modifiche involontarie al modello architetturale, un requisito critico in un contesto in cui il modello di dominio è l'artefatto centrale della pipeline.
\end{enumerate}

\section{Sviluppi Futuri}

\subsection{Evoluzione dello Stack di Osservabilità}

Lo stack attuale (OpenTelemetry Collector, Loki, Tempo, Prometheus, Grafana) è operativo ma può essere esteso con:
\begin{itemize}
  \item Dashboard Grafana personalizzate per il monitoraggio delle sessioni di discovery (latenza per fase, token consumati, tasso di successo della generazione JSON).
  \item Regole di alerting per anomalie (es. generazione JSON fallita, latenza superiore a soglia).
  \item Metriche custom esposte da n8n per l'analisi delle performance degli agenti.
\end{itemize}

\subsection{Validazione Automatica del JSON}

L'aggiunta di un nodo n8n di validazione post-generazione che verifica la conformità del JSON allo schema target, con meccanismo di retry automatico in caso di output malformato, migliorerebbe significativamente la robustezza della pipeline.

\subsection{Supporto Multi-Lingua}

L'estensione dei system prompt per supportare sessioni in lingue diverse dall'italiano (inglese, spagnolo, tedesco) amplierebbe l'applicabilità del DIM a contesti internazionali.

\subsection{Integrazione con il Pipeline Multi-Agente}

Il DIM è il primo agente di una pipeline più ampia. Gli sviluppi futuri prevedono l'integrazione con gli agenti downstream che consumano il JSON architetturale per:
\begin{itemize}
  \item Generazione automatica di diagrammi UML/C4 dai Bounded Context.
  \item Scaffolding del codice sorgente dei microservizi basato sul modello di dominio.
  \item Generazione di test di contratto (Contract Testing) basati sugli eventi DDD.
\end{itemize}

\subsection{Persistenza Long-Term e Versioning}

L'implementazione di un meccanismo di export/import delle sessioni (es. salvataggio su filesystem o database documentale) e di versioning del modello architetturale (storico delle modifiche con diff) consentirebbe la gestione di progetti di discovery complessi e multi-seduta.

\subsection{Miglioramenti al Frontend}

\begin{itemize}
  \item Visualizzazione grafica del JSON architetturale (tree view interattiva, diagramma dei Bounded Context).
  \item Progress bar della fase di discovery (indicatore visuale della fase corrente su 5).
  \item Supporto al dark mode.
  \item Export del modello in formati alternativi (YAML, PlantUML, Mermaid).
\end{itemize}
